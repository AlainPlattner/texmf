% \iffalse meta-comment
%
%   Copyright (C) 2004 by Sergey Fomel 
%  
%    This work may be distributed and/or modified under the
%    conditions of the LaTeX Project Public License, either version 1.3
%    of this license or (at your option) any later version.
%    The latest version of this license is in
%      http://www.latex-project.org/lppl.txt
%    and version 1.3 or later is part of all distributions of LaTeX
%    version 2003/12/01 or later.
%
% \fi
%
% \iffalse
%<class>\NeedsTeXFormat{LaTeX2e}
%<class>\ProvidesClass{segabs}[2005/04/03 v1.0 SEG expanded abstract]
%
%<*driver>
\ProvidesFile{segabs.dtx}[2005/04/03 v1.0 segabs class]
\documentclass{ltxdoc}
\GetFileInfo{segabs.dtx}
\EnableCrossrefs
\RecordChanges
\CodelineIndex
\begin{document}
\DocInput{segabs.dtx}
\end{document}
%</driver>
%\fi
%
% \CheckSum{0}
%% \CharacterTable
%%  {Upper-case    \A\B\C\D\E\F\G\H\I\J\K\L\M\N\O\P\Q\R\S\T\U\V\W\X\Y\Z
%%   Lower-case    \a\b\c\d\e\f\g\h\i\j\k\l\m\n\o\p\q\r\s\t\u\v\w\x\y\z
%%   Digits        \0\1\2\3\4\5\6\7\8\9
%%   Exclamation   \!     Double quote  \"     Hash (number) \#
%%   Dollar        \$     Percent       \%     Ampersand     \&
%%   Acute accent  \'     Left paren    \(     Right paren   \)
%%   Asterisk      \*     Plus          \+     Comma         \,
%%   Minus         \-     Point         \.     Solidus       \/
%%   Colon         \:     Semicolon     \;     Less than     \<
%%   Equals        \=     Greater than  \>     Question mark \?
%%   Commercial at \@     Left bracket  \[     Backslash     \\
%%   Right bracket \]     Circumflex    \^     Underscore    \_
%%   Grave accent  \`     Left brace    \{     Vertical bar  \|
%%   Right brace   \}     Tilde         \~}
%
% \changes{v1.0}{2005/04/03}{Initial version}
%
% \DoNotIndex{\LoadClass,\DeclareOption,\ProcessOption}
% \DoNotIndex{\@ifstar,\@mkboth,\@tempboxa,\\,\ ,\newcommand,
% \providecommand,\renewcommand}
%
% \title{The \textsf{segabs} class for 
% writing SEG expanded abstracts\thanks{This document corresponds to
% \textsf{segabs.cls}~\fileversion, dated~\filedate.}}
% \author{Sergey Fomel \\ \texttt{sergey.fomel@beg.utexas.edu}}
% \maketitle
% \begin{abstract}
% This package provides a \LaTeX2e\ class for writing SEG expanded abstracts.
% \end{abstract}
%
% \section{Introduction}
% \LaTeX\ is a powerful document typesetting system. The new
% \texttt{segabs.cls} package conforms to the \LaTeX2e\ standard.
%
% \section{Usage}
%
%\subsection{Options}
%
% Two options control the appearance...
%
%\subsection{Appearance}
%
% \DescribeEnv{abstract}
% The |abstract| environment...
%
% The following macros are defined in the package:
%
% \subsection{Figures and tables}
% \DescribeMacro{\inputdir}
% |\inputdir|\marg{dirname}
% specifies the top-level directory (current directory by default).
%
% \DescribeMacro{\figdir}
% |\figdir| stores the name of the figure directory relative to the directory
% specified by |\inputdir|. |\figdir| is ``|.|'' (the current directory) by
% default. You can change it with |\renewcommand{\figdir}|\marg{dirname}.
%
% \DescribeMacro{\plot}
% |\plot|\marg{filename}\marg{size}\marg{caption} inserts a figure from file
% \meta{filename} (specified relative to figdir) with caption
% \meta{caption}. The \meta{size} specification corresponds to the conventions
% of the \textsf{graphicx} package. Examples:
% \begin{itemize}
% \item |width=\textwidth| (for scaling the figure and preserving the aspect
% ratio)
% \item |width=3in,height=0.5\textheight| (for scaling the figure
% anisotropically) 
% \end{itemize}
% You can refer to the figure number with |\ref{fig:|\meta{filename}|}|.
%
% \DescribeMacro{\sideplot}
% |\sideplot|\marg{filename}\marg{size}\marg{caption} ...
%
% \DescribeMacro{\tabl}
% The |\tabl|\marg{label}\marg{caption}\marg{table} macro is used to 
% manage tables. In the manuscript mode, the defined \meta{table} will appear at% the end of the 
% paper, and its caption will be
% added to the list of tables. You can refer to the table number with 
% |\ref{tbl:|\meta{filename}|}|.
%
% \StopEventually{\PrintIndex\PrintChanges}
%
% \section{Implementation}
% Our package is a derivative of the standard \texttt{article} class.
%    \begin{macrocode}
\DeclareOption*{\PassOptionsToClass{\CurrentOption}{article}}
%    \end{macrocode}
% We will use 11pt fonts by default.
%    \begin{macrocode}
\ProcessOptions*
\LoadClass[11pt]{article}
%    \end{macrocode}
% \subsection{Fonts}
% Load fonts for text and math.
%    \begin{macrocode}
\usepackage{mathptmx}
\DeclareSymbolFont{largesymbols}{OMX}{cmex}{m}{n}
%    \end{macrocode}
%
% Load SEG style package
%    \begin{macrocode}
\RequirePackage{seg}
%    \end{macrocode}
%
% \subsection{Figures and Tables}
% \begin{macro}{\figdir}
% |\figdir| is a variable. It will need to be redefined to specify the figure
% directory. 
%    \begin{macrocode}
\providecommand{\figdir}{.}
\providecommand{\fig@file}[1]{\@path/\figdir/#1}
%    \end{macrocode}
% \end{macro}
% Count figures.
%    \begin{macrocode}
\newcounter{@plots}
\setcounter{@plots}{0}
%    \end{macrocode}
% Define commands for figure inclusion.
% \begin{macro}{\plot}
%    \begin{macrocode}
\DeclareRobustCommand{\fullplot}[4][htbp]{%
\begin{figure*}[#1]
\centering
\includegraphics[#3]{\fig@file{#2}}
\caption[]{#4}
\label{fig:#2}
\end{figure*}}
%    \end{macrocode}
% \end{macro}
% \begin{macro}{\sideplot}
%    \begin{macrocode}
\DeclareRobustCommand{\sideplot}[4][htbp]{%
\begin{figure}[#1]
\centering
\includegraphics[#3]{\fig@file{#2}}
\caption[]{#4}
\label{fig:#2}
\end{figure}}
%    \end{macrocode}
% \end{macro}
% \begin{macro}{\plot}
%    \begin{macrocode}
\def\plot{\@ifstar{\fullplot}{\sideplot}}
%    \end{macrocode}
% \end{macro}
% \begin{macro}{\multiplot}
%    \begin{macrocode}
\RequirePackage{subfigure}
\RequirePackage{ifthen}
\def\next@item#1,#2?{#1}
\def\rest@item#1,#2?{#2}
\newcounter{sub@fig}
%    \end{macrocode} 
%    \begin{macrocode}
\newcommand{\sidemultiplot}[5][htbp]{
  \begin{figure}[#1]
    \centering
    \setcounter{sub@fig}{0}
    \edef\list@i{#3}
    \loop
    \edef\item@i{\expandafter\next@item\list@i,\empty?}
    \edef\list@i{\expandafter\rest@item\list@i,\empty?}
    \ifx\item@i\empty 
    \else
    \stepcounter{sub@fig}
    \subfigure[]{\includegraphics[#4]{\fig@file{\item@i}}%
      \label{fig:\item@i}}
    \ifthenelse{\value{sub@fig} = #2}{\\ \setcounter{sub@fig}{0}}{}   
    \repeat
    \caption{#5}
    \label{fig:#3}
  \end{figure}
}
\newcommand{\fullmultiplot}[5][htbp]{
  \begin{figure*}[#1]
    \centering
    \setcounter{sub@fig}{0}
    \edef\list@i{#3}
    \loop
    \edef\item@i{\expandafter\next@item\list@i,\empty?}
    \edef\list@i{\expandafter\rest@item\list@i,\empty?}
    \ifx\item@i\empty 
    \else
    \stepcounter{sub@fig}
    \subfigure[]{\includegraphics[#4]{\fig@file{\item@i}}%
      \label{fig:\item@i}}
    \ifthenelse{\value{sub@fig} = #2}{\\ \setcounter{sub@fig}{0}}{}   
    \repeat
    \caption{#5}
    \label{fig:#3}
  \end{figure*}
}
\def\multiplot{\@ifstar{\fullmultiplot}{\sidemultiplot}}
%    \end{macrocode}
% \end{macro}
% \begin{macro}{\tabl}
% Define the |\tabl| macro for handling tables.
%    \begin{macrocode}
\providecommand{\tabl}[4][htbp]{
  \begin{table}[#1]
    #4
    \caption{#3}
    \label{tbl:#2}
  \end{table}
}
%    \end{macrocode}
% \end{macro}
% \subsection{References}
% Load the \textsf{natbib} package for natural-science-style citations.
%    \begin{macrocode}
\RequirePackage{natbib}
%    \end{macrocode}
% The following is \textsf{natbib}'s default.
%    \begin{macrocode}
\bibpunct[,]{(}{)}{;}{a}{,}{,}
%    \end{macrocode}
% Redefine the reference section name.
%    \begin{macrocode}
\DeclareRobustCommand{\refname}{REFERENCES}
\renewcommand\bibsection{\clearpage\section{\refname}}
%    \end{macrocode}
% For backward compatibility: shortcite refers to the year
%    \begin{macrocode}
\newcommand{\shortcite}[1]{\citeyearpar{#1}}
%    \end{macrocode}
%
% \subsection{Page layout}
% Specify page dimensions. \emph{This should be adjusted for A4 paper format!}
%    \begin{macrocode}
\setlength{\oddsidemargin}{0in}
\setlength{\evensidemargin}{0in}
\setlength{\textwidth}{6.5in}
\setlength{\textheight}{9.0625in}
\setlength{\parindent}{0em}
\setlength{\headheight}{0.3125in}
\setlength{\headsep}{0.125in}
\setlength{\parskip}{1.5ex plus0.1ex minus0.1ex}
\setlength{\topsep}{0em}%
\setlength{\itemsep}{0em}
\setlength{\topmargin}{-0.5in}
\setlength{\columnsep}{0.3125in}
%    \end{macrocode}
% \subsection{Font size}
% Messing with sizes to get the required 9pt Times-Roman
%    \begin{macrocode}
\let\normal@size\normalsize
\let\seg@small\small
\let\seg@size\footnotesize
\renewcommand{\Large}{\protect\seg@size}
\renewcommand{\normalsize}{\protect\seg@size}
\renewcommand{\small}{\protect\tiny}
\renewcommand{\footnotesize}{\protect\tiny}
\seg@size
%    \end{macrocode}
% \subsection{Section commands}
%    \begin{macrocode}
\newcommand{\seg@section}{%
\@startsection {section}{1}{\z@}%
{-3.5ex \@plus -1ex \@minus -.2ex}%
{1ex \@plus .2ex}%
{\normalfont\Large\bfseries\MakeUppercase}}
\newcommand{\seg@subsection}[2][]{%
\begin{minipage}{\columnwidth}
\textbf{#2} 
\end{minipage}}
\newcommand{\seg@subsubsection}[2][]{%
\begin{minipage}{\columnwidth}
\textit{#2} 
\end{minipage}}
\newcommand{\seg@subsubsubsection}[2][]{%
\underline{#2}.--}
\def\section{\@ifstar{\seg@section*}{\seg@section*}}%
\def\subsection{\@ifstar{\seg@subsection}{\seg@subsection}}%
\def\subsubsection{\@ifstar{\seg@subsubsection}{\seg@subsubsection}}
\def\subsubsubsection{\@ifstar{\seg@subsubsubsection}{\seg@subsubsubsection}}
%    \end{macrocode}
% \subsection{Title and author}
% \begin{macro}{\title}
% |\title| is redefined for consistency. It changes the internal |\seg@title|.
%    \begin{macrocode}
\newcommand{\seg@title}{}
\renewcommand{\title}[1]{\renewcommand{\seg@title}{#1}}
%    \end{macrocode}
% \end{macro}
% \begin{macro}{\address}
% The |\address| macro will save the address in the internal |\seg@address|.
%    \begin{macrocode}
\providecommand{\seg@address}{}
\providecommand{\address}[1]{\renewcommand{\seg@address}{\emph{#1}}}
%    \end{macrocode}
% \end{macro}
% \begin{macro}{\email}
% The |\email| macro will save the e-mail address in the internal 
% |\seg@email|.
%    \begin{macrocode}
\providecommand{\seg@email}{.}
\providecommand{\email}[1]{\renewcommand{\seg@email}{#1}}
%    \end{macrocode}
% \end{macro}
% \begin{macro}{\author}
% Finally, everything goes to \author and is stored in |\seg@author|.
%    \begin{macrocode}
\providecommand{\seg@author}{}
\renewcommand{\author}[1]{%
\renewcommand{\seg@author}{%
\twocolumn[\vspace{0.25in}\normal@size\textbf{\seg@title}%
\newline\seg@small\textit{#1}\vspace{2ex}]}}
%    \end{macrocode}
% \end{macro}
% \begin{macro}{\maketitle}
% Redefined |\maketitle| is dumping the |\seg@author|.
%    \begin{macrocode}
\renewcommand{\maketitle}{\seg@author}
%    \end{macrocode}
% \end{macro}
% \begin{macro}{abstract}
% The abstract environment produces a summary section.
%    \begin{macrocode}
\renewenvironment{abstract}{%
\begin{minipage}[t]{\columnwidth}
\textbf{SUMMARY\\}\\*}{%
\end{minipage}\vspace{2ex}}
%    \end{macrocode}
% \end{macro}
% \subsection{Page style}
% Define a page style
%    \begin{macrocode}
\newcommand{\ps@seg}{%
\renewcommand{\@oddhead}{\hfill\seg@small\textbf{\seg@rhead}\seg@size\hfill}
\renewcommand{\@evenhead}{\@oddhead}
\renewcommand{\@oddfoot}{}
\renewcommand{\@evenfoot}{}}
%
\thispagestyle{empty}
\pagestyle{seg}
%    \end{macrocode}
% \subsection{Appendix}
% Clean this mess later.
%    \begin{macrocode}
\newcounter{@append}
\providecommand{\append@name}{A}
\providecommand{\appendname}[1]{\renewcommand{\append@name}{#1}}
\renewcommand{\appendix}{%
  \ifthenelse{\equal{\append@name}{A}}{%
    \renewcommand{\append@name}{\Alph{@append}}}{}
  \setcounter{equation}{0}%
  \renewcommand{\theequation}{\mbox{\append@name-\arabic{equation}}}%
    \setcounter{figure}{0}%
    \renewcommand{\thefigure}{\append@name-\arabic{figure}}%
  \stepcounter{@append}
}
\providecommand{\append}[1]{%
\appendix\section{Appendix \append@name}\section{#1}}
%    \end{macrocode}
% Packages for including things verbatum.
%    \begin{macrocode}
\RequirePackage{verbatim}
\RequirePackage{moreverb}
%    \end{macrocode}
% \Finale
\endinput
