% \iffalse meta-comment
%
%   Copyright (C) 2004 by Sergey Fomel 
%  
%    This work may be distributed and/or modified under the
%    conditions of the LaTeX Project Public License, either version 1.3
%    of this license or (at your option) any later version.
%    The latest version of this license is in
%      http://www.latex-project.org/lppl.txt
%    and version 1.3 or later is part of all distributions of LaTeX
%    version 2003/12/01 or later.
%
% \fi
%
% \iffalse
%<class>\NeedsTeXFormat{LaTeX2e}
%<class>\ProvidesClass{geophysics}[2004/06/22 v1.0 Geophysics paper]
%
%<*driver>
\ProvidesFile{geophysics.dtx}[2004/06/22 v1.0 geophysics class]
\documentclass{ltxdoc}
\GetFileInfo{geophysics.dtx}
\EnableCrossrefs
\RecordChanges
\CodelineIndex
\begin{document}
\DocInput{geophysics.dtx}
\end{document}
%</driver>
%\fi
%
% \CheckSum{0}
%% \CharacterTable
%%  {Upper-case    \A\B\C\D\E\F\G\H\I\J\K\L\M\N\O\P\Q\R\S\T\U\V\W\X\Y\Z
%%   Lower-case    \a\b\c\d\e\f\g\h\i\j\k\l\m\n\o\p\q\r\s\t\u\v\w\x\y\z
%%   Digits        \0\1\2\3\4\5\6\7\8\9
%%   Exclamation   \!     Double quote  \"     Hash (number) \#
%%   Dollar        \$     Percent       \%     Ampersand     \&
%%   Acute accent  \'     Left paren    \(     Right paren   \)
%%   Asterisk      \*     Plus          \+     Comma         \,
%%   Minus         \-     Point         \.     Solidus       \/
%%   Colon         \:     Semicolon     \;     Less than     \<
%%   Equals        \=     Greater than  \>     Question mark \?
%%   Commercial at \@     Left bracket  \[     Backslash     \\
%%   Right bracket \]     Circumflex    \^     Underscore    \_
%%   Grave accent  \`     Left brace    \{     Vertical bar  \|
%%   Right brace   \}     Tilde         \~}
%
% \changes{v1.0}{2004/06/22}{Initial version}
%
% \DoNotIndex{\LoadClass,\DeclareOption,\ProcessOption}
% \DoNotIndex{\newcommand,\providecommand,\renewcommand}
%
% \title{The \textsf{geophysics} class for 
% writing \emph{Geophysics} papers\thanks{This document corresponds to
% \textsf{geophhysics}~\fileversion, dated~\filedate.}}
% \author{Sergey Fomel \\ \texttt{sergey.fomel@beg.utexas.edu}}
% \maketitle
% \begin{abstract}
% This package provides...
% \end{abstract}
%
% \section{Introduction}
% \LaTeX\ is a powerful system... 
%
% \section{Usage}
%
% The following macros are defined in the package:
%
% \subsection{SEG-style vectors and tensors}
% \DescribeMacro{\vector}
% |\vector|\marg{v} is used for vector variables in equations. It simply puts
% the argument in boldface.
%
% \DescribeMacro{\tensor}
% |\tensor|\marg{t} is used for matrix variables in equations. It puts the
% argument in boldface and underscores it with a tilde.
%
% \DescribeEnv{abstract}
% Put an abstract
%
% \subsection{Figures and tables}
% \DescribeMacro{\inputdir}
% |\inputdir|\marg{dirname}
% specifies the top-level directory (current directory by default).
%
% \DescribeMacro{\figdir}
% |\figdir| stores the name of the figure directory relative to the directory
% specified by |\inputdir|. |\figdir| is ``|.|'' (the current directory) by
% default. You can change it with |\renewcommand{\figdir}|\marg{dirname}.
%
% \DescribeMacro{\plot}
% |\plot|\marg{filename}\marg{size}\marg{caption} inserts a figure from file
% \meta{filename} (specified relative to figdir) with caption
% \meta{caption}. The \meta{size} specification corresponds to the convention
% of the \textsf{graphicx} package. Examples:
% \begin{itemize}
% \item |width=\textwidth| (for scaling the figure and preserving the aspect
% ratio)
% \item |width=3in,height=0.5\textheight| (for scaling the figure
% anisotropically) 
% \end{itemize}
% Refer to the figure number with |\ref{fig:|\meta{filename}|}|.
%
% \DescribeMacro{\tabl}
% The |\tabl|\marg{label}\marg{caption}\marg{table} macro is used to 
% manage tables. The defined \meta{table} will appear at the end of the 
% paper, and its caption will be
% added to the list of tables. Refer to the table number with 
% |\ref{tbl:|\meta{filename}|}|.
%
% \StopEventually{\PrintIndex\PrintChanges}
%
% \section{Implementation}

% Load \textsf{ifthen} package for conditionals.
%    \begin{macrocode}
\RequirePackage{ifthen}
%    \end{macrocode}
% Different types: can prepare a manuscript, a paper, or a book.
%    \begin{macrocode}
\newcommand{\@type}{}
\DeclareOption{manuscript}{\renewcommand{\@type}{manuscript}}
\DeclareOption{paper}{\renewcommand{\@type}{paper}}
\DeclareOption{book}{\renewcommand{\@type}{book}}
%    \end{macrocode}
% Our package is a derivative of the standard \texttt{article} class.
%    \begin{macrocode}
\DeclareOption*{\PassOptionsToClass{\CurrentOption}{article}}
%    \end{macrocode}
% We will use paper and 11pt fonts by default.
%    \begin{macrocode}
\ExecuteOptions{paper}
\ProcessOptions*
\LoadClass[11pt]{article}
%    \end{macrocode}
%
% \subsection{SEG-style vectors and tensors}
% \begin{macro}{\vector}
% The SEG-style vector is simply bold math.
%    \begin{macrocode}
\renewcommand{\vector}[1]{\ensuremath{\mathbf{#1}}}
%    \end{macrocode}
% \end{macro}
% \begin{macro}{\tensor}
% The SEG-style matrix is tilde-underscored vector.
%    \begin{macrocode}
\newcommand{\under@tilde}[1]{\mathord{\vtop{\ialign{##\crcr
        $\hfil\displaystyle{#1}\hfil$\crcr\noalign{\kern1.5pt\nointerlineskip}
        $\hfil\tilde{}\hfil$\crcr\noalign{\kern1.5pt}}}}}
\providecommand{\tensor}[1]{\ensuremath{\under@tilde{\vector{#1}}}}
%    \end{macrocode}
% \end{macro}
%
% \subsection{Figures and Tables}
% \begin{macro}{\@path}
% |\@path| is an internal variable to store the top directory.
% \begin{macro}{\inputdir}
%    \begin{macrocode}
\newcommand{\@path}{.}
\newcommand{\inputdir}[1]{\renewcommand{\@path}{#1}}
%    \end{macrocode}
% \end{macro}
% \end{macro}
% \begin{macro}{\figdir}
% |\figdir| is a variable. It will need to be redefined to specify the figure
% directory. 
%    \begin{macrocode}
\providecommand{\figdir}{.}
\providecommand{\fig@file}[1]{\@path/\figdir/#1}
%    \end{macrocode}
% \end{macro}
% Load the \textsf{graphicx} package for included figures.
%    \begin{macrocode}
\RequirePackage{graphicx}
%    \end{macrocode}
% Possible suffices for graphics files.
%    \begin{macrocode}
\DeclareGraphicsExtensions{.pdf,.eps,.ps}
%    \end{macrocode}
% Count figures.
%    \begin{macrocode}
\newcounter{@plots}
\setcounter{@plots}{0}
%    \end{macrocode}
% \begin{macro}{\plot}
% Define commands for figure inclusion.
%    \begin{macrocode}
\ifthenelse{\equal{\@type}{manuscript}}
%    \end{macrocode} 
% In manuscripts, each figure appears on a separate page at the
% end of the paper.
%    \begin{macrocode}
{\newcommand{\plot}[4][X]{
    \stepcounter{@plots}
    \AtEndDocument{
      \begin{figure}[p]
        \centering
        \includegraphics[#3]{\fig@file{#2}}
        \caption[#4]{}
        \label{fig:#2}
      \end{figure}\clearpage
    }}
  \long\def\@makecaption#1#2{%
    \vskip\abovecaptionskip
    \sbox\@tempboxa{#1.}%
    \ifdim \wd\@tempboxa >\hsize
    #1.\par
    \else
    \global \@minipagefalse
    \hb@xt@\hsize{\hfil\box\@tempboxa\hfil}%
    \fi
    \vskip\belowcaptionskip}}
%    \end{macrocode}
%In papers, an optional argument places the figure.
%    \begin{macrocode}
{\newcommand{\plot}[4][htbp]{%
    \begin{figure*}[#1]
      \centering
      \includegraphics[#3]{\fig@file{#2}}
      \caption{#4}
      \label{fig:#2}
    \end{figure*}
  }}
%    \end{macrocode} 
% \end{macro}
% \begin{macro}{\sideplot}
% Unless in manuscript mode, |\sideplot| puts figures on the side 
%    \begin{macrocode}
\ifthenelse{\equal{\@type}{manuscript}}
%    \end{macrocode} 
% In manuscripts, there is no difference.
%    \begin{macrocode}
{\newcommand{\sideplot}[4][X]{\plot[#1]{#2}{#3}{#4}}}
%In papers, an optional argument places the figure. The page is split in half
%to put the figure on the side.
%    \begin{macrocode}
{\newcommand{\sideplot}[4][htbp]{%
    \begin{figure*}[#1]
      \makebox{%
        \begin{minipage}{0.4\textwidth}
          \caption{#4}
          \label{fig:#2}
        \end{minipage}
        \hfill
        \begin{minipage}{0.6\textwidth}
          \centering
          \includegraphics[#3]{\fig@file{#2.pdf}}
        \end{minipage}}
    \end{figure*}
  }}
%    \end{macrocode} 
% \end{macro}
% Count tables.
%    \begin{macrocode}
\newcounter{@tabls}
\setcounter{@tabls}{0}
%    \end{macrocode}
% \begin{macro}{\tabl}
% Define the |\tabl| macro for handling tables.
%    \begin{macrocode}
\providecommand{\tabl}[4][X]{
  \stepcounter{@tabls}
  \AtEndDocument{
    \begin{table}[p]
      #4
      \caption[#3]{}
      \label{tbl:#2}
    \end{table}\clearpage
  }
}
%    \end{macrocode}
% \end{macro}
% Suppress page numbers from lists of figures
%    \begin{macrocode}
\renewcommand*{\l@figure}[2]{%
  \setlength{\@tempdima}{2.3em}%
  \noindent\hspace*{1.5em}#1\hfil\newline}
\let\l@table\l@figure
%    \end{macrocode}
% Redefine lists of figures and tables
%    \begin{macrocode}
\renewcommand\listoffigures{%
  \section{\listfigurename
    \@mkboth{\MakeUppercase\listfigurename}%
    {\MakeUppercase\listfigurename}}%
  \@starttoc{lof}%
}
\renewcommand\listoftables{%
  \section{\listtablename
    \@mkboth{\MakeUppercase\listtablename}%
    {\MakeUppercase\listtablename}}%
  \@starttoc{lot}%
}
%    \end{macrocode}
% List of figures and list of tables appear at the end of the document.
%    \begin{macrocode}
\AtEndDocument{
  \ifthenelse{\value{@tabls} > 0}{\newpage\listoftables}{}
  \ifthenelse{\value{@plots} > 0}{\newpage\listoffigures}{}
}
%    \end{macrocode}
% \subsection{References}
% Load the \textsf{natbib} package for natural-science-style citations.
%    \begin{macrocode}
\RequirePackage{natbib}
%    \end{macrocode}
% The following is \textsf{natbib}'s default.
%    \begin{macrocode}
\bibpunct[,]{(}{)}{;}{a}{,}{,}
%    \end{macrocode}
% Redefine the reference section name.
%    \begin{macrocode}
\DeclareRobustCommand{\refname}{REFERENCES}
\ifthenelse{\equal{\@type}{manuscript}}
%    \end{macrocode}
% In manuscripts, references appear at a separate page
%    \begin{macrocode}
{\renewcommand\bibsection{\newpage\section{\refname}}}
{\renewcommand\bibsection{\section{\refname}}}
%    \end{macrocode}
% For backward compatibility: shortcite refers to the year
%    \begin{macrocode}
\newcommand{\shortcite}[1]{\citeyearpar{#1}}
%    \end{macrocode}
%
% \subsection{Fonts}
% Load postscript fonts for text and math.
%    \begin{macrocode}
\RequirePackage{mathptmx}
%    \end{macrocode}
%
% \subsection{Page layout}
% Do double spacing for manuscripts.
%    \begin{macrocode}
\ifthenelse{\equal{\@type}{manuscript}}{
  \renewcommand{\baselinestretch}{2.0}
}{
  \renewcommand{\baselinestretch}{1.0}
}
%    \end{macrocode}
% Specify page dimensions. \emph{This should be adjusted for A4 paper format.}
%    \begin{macrocode}
\setlength{\textheight}{9in}
\setlength{\textwidth}{6in}
\setlength{\parskip}{1.5ex plus0.1ex minus0.1ex} % should be a *rubber* length
\setlength{\oddsidemargin}{.26in}
\setlength{\evensidemargin}{.26in}
\setlength{\topmargin}{-.51in}
\setlength{\textfloatsep}{30pt plus 3pt minus 6pt}
\setlength{\headsep}{0.5in}
%    \end{macrocode}
% \subsection{Section headings}
% Capitalize section headings and redefine their size from 
% |\Large| to |\large|%
%\footnote{Thanks to Joerg Schleicher for the ifstar trick.}.
%    \begin{macrocode}
\let\seg@large\large
\let\seg@Large\Large
\renewcommand{\Large}{\protect\seg@large}
\let\seg@section\section
\let\seg@subsection\subsection
\let\seg@subsubsection\subsubsection
\newcommand{\segsection}[2][]{\seg@section*{\MakeUppercase{#2}}}
\def\section{\@ifstar{\segsection}{\segsection}}
\def\subsection{\@ifstar{\seg@subsection*}{\seg@subsection*}}%
\def\subsubsection{\@ifstar{\seg@subsubsection*}{\seg@subsubsection*}}
%    \end{macrocode}
% \subsection{Title and author}
% \begin{macro}{\title}
% |\title| is redefined for consistency. It changes the internal |\seg@title|.
%    \begin{macrocode}
\newcommand{\seg@title}{}
\newenvironment{geo@title}{%
\vspace{.3in}\begin{minipage}{\textwidth}%
\bfseries\seg@Large\centering}{\end{minipage}\vspace{.2in}}
\renewcommand{\title}[1]{%
\renewcommand{\seg@title}{%
\noindent\begin{geo@title}\setlength{\parindent}{0em}\par%
{\smallskip #1}\end{geo@title}}}
%    \end{macrocode}
% \end{macro}
% \begin{macro}{\address}
% The |\address| macro will save the address in the internal |\seg@address|.
%    \begin{macrocode}
\providecommand{\seg@address}{}
\providecommand{\address}[1]{\renewcommand{\seg@address}{\emph{#1}}}
%    \end{macrocode}
% \end{macro}
% \begin{macro}{\email}
% The |\email| macro will save the e-mail address in the internal 
% |\seg@email|.
%    \begin{macrocode}
\providecommand{\seg@email}{.}
\providecommand{\email}[1]{\renewcommand{\seg@email}{#1}}
%    \end{macrocode}
% \end{macro}
% \begin{macro}{\author}
% Finally, everything goes to \author and is stored in |\seg@author|.
%    \begin{macrocode}
\providecommand{\seg@author}{}
\renewcommand{\author}[1]{%
  \renewcommand{\seg@author}{%
    \seg@title
      \begin{centering}
        \ifthenelse{\equal{\@type}{manuscript}}{%
          \textbf{#1} \\
          \seg@address\ \\
          (\today) \\
        }{\textit{#1}\ifthenelse{\equal{\seg@email}{.}}{}{%
            \footnote{\textbf{e-mail: }\seg@email}} \\ \ 
        }
      \end{centering}
    }}
%    \end{macrocode}
% \end{macro}
% \begin{macro}{\maketitle}
% Redefined |\maketitle| is dumping the |\seg@author|.
%    \begin{macrocode}
\renewcommand{\maketitle}{\seg@author}
%    \end{macrocode}
% \end{macro}
% \begin{macro}{abstract}
% For manuscripts, the abstract environment is simply a section.
%    \begin{macrocode}
\ifthenelse{\equal{\@type}{manuscript}}{
  \renewenvironment{abstract}{\section{ABSTRACT}}{\newpage}
}
%    \end{macrocode}
% For papers, the abstract apears in a box.
%    \begin{macrocode}
{
  \newsavebox{\seg@abstract}
  \setlength{\fboxsep}{10pt}
  \renewenvironment{abstract}{
    \begin{lrbox}{\seg@abstract}\begin{minipage}{0.95\textwidth}%
        \centerline{\textbf{ABSTRACT}}\medskip}{%
        \medskip\end{minipage}\end{lrbox}%
    \noindent\fbox{\usebox{\seg@abstract}}
  }
} 
%    \end{macrocode}
% \end{macro}
% \subsection{Headers and footers}
% Internal storage
%    \begin{macrocode}
%    \end{macrocode}
% \begin{macro}{\righthead}
% Right header
%    \begin{macrocode}
\newcommand{\seg@rhead}{}
\newcommand{\righthead}[1]{\renewcommand{\seg@rhead}{#1}}
%    \end{macrocode}
% \end{macro}
% \begin{macro}{\lefthead}
% Left header
%    \begin{macrocode}
\newcommand{\seg@lhead}{}
\newcommand{\lefthead}[1]{\renewcommand{\seg@lhead}{#1}}
%    \end{macrocode}
% \end{macro}
% \begin{macro}{\footer}
% Footer
%    \begin{macrocode}
\newcommand{\seg@foot}{}
\newcommand{\footer}[1]{\renewcommand{\seg@foot}{#1}}
%    \end{macrocode}
% \end{macro}
% \begin{macro}{\header}
% Header
%    \begin{macrocode}
\newcommand{\seg@head}{}
\newcommand{\header}[1]{\renewcommand{\seg@head}{#1}}
%    \end{macrocode}
% \end{macro}
% \subsection{Page style}
%    \begin{macrocode}
\ifthenelse{\equal{\@type}{manuscript}}
%    \end{macrocode}
% For manuscript, use plain page style.
%    \begin{macrocode}
{\pagestyle{plain}\thispagestyle{plain}}
%    \end{macrocode}
% For paper, define a page style
%    \begin{macrocode}
{\newcommand{\ps@seg}{%
\renewcommand{\@oddhead}{%
\makebox[2.5in][l]{\textit{\seg@lhead}}\hfill\thepage\hfill%
\makebox[2.5in][r]{\textit{\seg@rhead}}}%
\renewcommand{\@evenhead}{\@oddhead}%
\renewcommand{\@oddfoot}{\hfill\textit{\seg@foot}\hfill}%
\renewcommand{\@evenfoot}{\@oddfoot}}
\newcommand{\ps@segone}{%
\renewcommand{\@oddhead}{}%
\renewcommand{\@evenhead}{}%
\renewcommand{\@oddfoot}{\hfill\textit{\seg@foot}\hfill}%
\renewcommand{\@evenfoot}{\@oddfoot}}
\pagestyle{seg}\thispagestyle{segone}}
%    \end{macrocode}
% \subsection{Appendix}
% Clean this mess later.
%    \begin{macrocode}
\newcounter{@append}
\providecommand{\append@name}{A}
\providecommand{\appendname}[1]{\renewcommand{\append@name}{#1}}
\renewcommand{\appendix}{%
\ifthenelse{\equal{\append@name}{A}}{%
\renewcommand{\append@name}{\Alph{@append}}}{}
\setcounter{equation}{0}\setcounter{figure}{0}%%
\renewcommand{\theequation}{\mbox{\append@name-\arabic{equation}}}%
\renewcommand{\thefigure}{\append@name-\arabic{figure}}\stepcounter{@append}}
\providecommand{\APPENDIX}[1]{\appendname{#1}\appendix\section{APPENDIX\ #1}}
%    \end{macrocode}
% Packages for including things verbatum.
%    \begin{macrocode}
\RequirePackage{verbatim}
\RequirePackage{moreverb}
%    \end{macrocode}
% \Finale
\endinput